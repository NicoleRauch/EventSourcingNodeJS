%%%%%%%%%%%%%%%%%%%%%%%%%%%%%%%%%%%%%%%%%%%%%%%%%%%%%%%%%
% Progress Bar:
% http://tex.stackexchange.com/questions/59742/progress-bar-for-latex-beamer

\usetikzlibrary{calc}

\definecolor{pbgray}{HTML}{575757}% background color for the progress bar

\makeatletter
\def\progressbar@progressbar{} % the progress bar
\newcount\progressbar@tmpcounta% auxiliary counter
\newcount\progressbar@tmpcountb% auxiliary counter
\newdimen\progressbar@pbht %progressbar height
\newdimen\progressbar@pbwd %progressbar width
\newdimen\progressbar@tmpdim % auxiliary dimension

\progressbar@pbwd=\linewidth
\progressbar@pbht=1pt

% the progress bar
\def\progressbar@progressbar{%

    \progressbar@tmpcounta=\insertframenumber
    \progressbar@tmpcountb=\inserttotalframenumber
    \progressbar@tmpdim=\progressbar@pbwd
    \multiply\progressbar@tmpdim by \progressbar@tmpcounta
    \divide\progressbar@tmpdim by \progressbar@tmpcountb

  \begin{tikzpicture}[very thin]
    \draw[pbgray!30,line width=\progressbar@pbht]
      (0pt, 0pt) -- ++ (\progressbar@pbwd,0pt);
    \draw[draw=none]  (\progressbar@pbwd,0pt) -- ++ (2pt,0pt);
    \draw[fill=pbgray!30,draw=pbgray] %
       ( $ (\progressbar@tmpdim, \progressbar@pbht) + (0,1.5pt) $ ) -- ++(60:3pt) -- ++(180:3pt) ;

    \node[draw=pbgray!30,text width=3.5em,align=center,inner sep=1pt,
      text=pbgray!70,anchor=east] at (0,0) {\insertframenumber/\inserttotalframenumber};
  \end{tikzpicture}%
}

\addtobeamertemplate{headline}{}
{%
  \begin{beamercolorbox}[wd=\paperwidth,ht=5ex,center,dp=1ex]{white}%
    \progressbar@progressbar%
  \end{beamercolorbox}%
}
\makeatother

%%%%%%%%%%%%%%%%%%%%%%%%%%%%%%%%%%%%%%%%%%%%%%%%%%%%%%%%%
